\documentclass[12pt]{report}
\usepackage{amsmath}
\usepackage{amssymb}
\usepackage{ragged2e}
\usepackage{graphicx}
\usepackage{listings}

\newcommand{\Lim}[1]{\raisebox{0.5ex}{\scalebox{0.8}{$\displaystyle \lim_{#1}\;$}}}
\newcommand{\MLim}{\Lim{x\rightarrow\infty}}

\newcommand{\no}{\noindent}
\newcommand{\tab}{\hspace*{.6cm}}

\begin{document}
	
	\Large
	\centering
	CSE373 Homework 2
	
	\justify
	\normalsize
	
	Kuba Gasiorowski\\
	ID: 109776237\\
	
	\no \textbf{1. } Given two nodes $s$ and $t$, we want to find the highest probability path from $s$ to $t$. The algorithm will look like the following:
	
	\begin{enumerate}
		\item Create a topological ordering of the graph. Let $p(n)$ denote the highest probability path currently computed for a node $n$. Set $p(s) = 1$ and all other nodes $n \in G$, $p(n) = 0$.
		\item In the topological order obtained from (1) starting with $s$, do the following with each node $v$:
		\begin{enumerate}
			\item For every node $u$ that is adjacent to $v$, if $p(u) < p(v) \cdot p(u,v)$, then set $p(u) = p(v) \cdot p(u,v)$.
		\end{enumerate}
		\item Return $p(t)$.
	\end{enumerate}
	
	\no Note that this algorithm is basically exactly the same as finding the shortest path in a DAG. The only differences are that we are looking for the longest path (since we can treat probabilities as edge weights) and instead of adding probabilities, we must multiply them (since that's what you do with compounded probabilities, unlike path lengths).\\
	
	\no \textbf{2.} Since there should not be any cycles (A can't be a prerequisite of B if B is a prerequisite of A; and if A is required by B, and B is required by C, it would make no sense for C to be required by A) and prerequisites are unidirectional, this graph will be a DAG (directed acyclic graph). Finding the minimum number of semesters can be likened to finding the longest path in the DAG, as each node will have to be taken in a different semester. We define the algorithm for this as follows: 
	
	\begin{enumerate}
		\item Assume our graph is called $G$. Keep a counter $c_n$ for every node $n$, initialize each node's counter to the number of prerequisites that that node has. This is done in linear time. Also, keep an overall counter $C$ which is initialized to zero.
		\item While $G$ is not empty:
		\begin{enumerate}
			\item Delete all nodes that have $c_n = 0$ from $G$. This simulates "scheduling" a set of classes for one semester.
			\item For every node deleted, decrement all counters of nodes which have the recently deleted node as a prerequisite. This is also done linearly.
			\item Increment $C$.
		\end{enumerate}
		\item Return $C$.
	\end{enumerate} 

	\no This algorithm will return the longest path in the DAG, in other words, the number of semesters necessary to complete the degree. Since all the steps are linear, the overall algorithm also has linear time complexity.\\

	\no \textbf{3.} Let us define a function $t(a, b) =$ the time taken to travel from point $a$ to point $b$. We are trying to figure out an algorithm to find $t(SF, SB)$. 
	
	Consider $i$ stations that are within 100 miles of San Francisco (denoted as stations $x_1, x_2, \ldots x_i$). We must visit at least one of these stations, and so we must choose. 
	\begin{itemize}
		\item If we choose station $x_1$, then we get $t(SF,SB) = c_1 + t(x_1,SB)$
		\item If we choose station $x_2$, then we get $t(SF,SB) = c_2 + t(x_2,SB)$
		\item If we choose station $x_k$, then we get $t(SF,SB) = c_i + t(x_i,SB)$
	\end{itemize}

	Notice also that if we choose station $x_1$, then our range increases to include some extra set of stations, say $x_{i+j}$. If we then stop at $x_{k+j}$, then our time becomes $t(SF,SB) = c_1 + c_{i+j} + t(x_{i+j},SB)$. Now we can see the recursive solution: $\boxed{t(SF,SB) = min\{c_k + t(x_k,SB)\},\;1 \leq k \leq i}$ where $i$ is the number of stations within 100 miles of San Francisco. For the last station $x_n, \; t(x_n,SB) = 0$.\\

	\pagebreak

	\no The pseudocode is given below.
	
	\begin{lstlisting}
let dist be the array of distances from SF to each station 
let T be an array of size n, initialized to -1
T[n] = 0 (base case)

Time(t):

	if T[t] is not -1:
		return T[t]
	else:
		
		smallest = infinity
		counter = 1
		
		while dist[t + counter] - dist[t] < 100:
		
			temp = T[t+counter]+Time(t+counter)
			
			if temp < smallest:
				
				smallest = temp
				
			endif
			
			counter = counter + 1
			
		end while
		
		T[t] = smallest
		
		return T[t]
		
	end if
	\end{lstlisting}

	\no \textbf{4.} Not attempted.\\
	
	\pagebreak
	
	\no \textbf{5a.} We define some variables. We have convoy $C$ which has $i$ ships in it. Each ship's value is defined as $v_n\text{, where }1 \leq n \leq i.$ We define $b(C,n)$ as breaking convoy $C$ at ship $n$. We can easily define the recurrence relation as $b(C,n) = v_n + b(C,n-1)\; , 1 < n \leq i$, since breaking off the convoy at some ship $n$ is the same as breaking off the convoy one ship before in the convoy and adding the next ship's worth to that. Our base case is when $n=1$, or when Jack simply breaks off the first ship in the convoy. This is $b(C,1) = v_1$. Since this is a single-recursive relation, it's clear that the algorithm to actually solve it must be $O(n)$ complexity.\\

	\no \textbf{b.} This problem is similar to the one above. The still relation still holds, but we have to add an additional recursive relation to it. For every $n,\;1 \leq n \leq i$, we must consider that Jack's partner will break off the convoy a second time at some point $j,\; n < j \leq i$. So we redefine our function as $b(C, n, j) = v_n + b(C,n-1,j) - b(C,n,j+1)$. We have two base cases: when $j=i$ and when $n=1$. When $j = i$, return $v_i$. When $n = 0$, return $v_1$. Since this is doubly-recursive we can use memoization to skip repetitive computations. Since we need to recursively find the sum of ships after the second break-off point multiple times per call, we can store the sums in an array and simply recall them when they are needed.

\end{document}
